
\documentclass{beamer}
\usepackage[spanish]{babel}
\usepackage[latin1]{inputenc}
\usepackage{multicol} % indice en 2 columnas

\usetheme{Warsaw}
\usecolortheme{crane}
\useoutertheme{shadow}
\useinnertheme{rectangles}

\setbeamertemplate{navigation symbols}{} % quitar simbolitos

\title[Algoritmo de Hashing]{Algoritmo de Hashing}
\subtitle{B�squeda}
\author[Mart�nez, Canales, Reyes, Rubilar]
{D. Mart�nez \\
F. Canales \\
J. Reyes \\
F. Rubilar}
\institute[EDEN \& HELL]
{
  Universidad Tecnol�gica Metropolitana\\
  Profesor: Alejandro Reyes
}

\AtBeginSection{

  \tableofcontents[currentsection]  
}

\AtBeginSubsection{
  \tableofcontents[currentsection,currentsubsection]

}

\begin{document}

\frame{\titlepage}

\begin{frame}
  \frametitle{Introducci�n}
  En esta presentaci�n daremos a conocer el funcionamiento y las caracter�ticas
	que posee la funci�n Hash. Dejando en claro sus ventajas, desventajas y las
	propiedades de �sta.
\end{frame}

\begin{frame}
  \frametitle{Funci�n Hash}
  Es una funci�n para resumir o identificar probabil�sticamente un gran
	conjunto de informaci�n, dando como resultado un conjunto imagen finito.
	Algunas de las funciones hash m�s utilizadas son las
	siguientes:
	\begin{itemize}
  \item<1->{Funci�n Cuadrada}
  \item<2->{Funci�n de Plegamiento}
  \item<3->{Funci�n de Truncamiento}
	\end{itemize}
\end{frame}

\begin{frame}
  \frametitle{Propiedades de Funci�n Hash}
	\begin{itemize}
  \item<1->{Bajo Costo}
  \item<2->{Compresi�n}
  \item<3->{Uniformidad}
	\item<4->{Rango Variable}
  \item<5->{Inyectividad}
  \item<6->{Determinismo}
  \end{itemize}
\end{frame}

\begin{frame}
  \frametitle{Ventajas y Desventajas}
  \begin{block}{Ventajas}
  Se pueden usar los valores naturales de la llave, puesto que se traducen
	internamente a direcciones f�ciles de localizar Se logra independencia
	l�gica y f�sica, debido a que los valores de las llaves son
	independientes del espacio de direcciones No se requiere almacenamiento
	adicional para los �ndices.
  \end{block}
      
  \begin{block}{Desventajas}
  El archivo no esta clasificado No permite llaves repetidas Solo permite
	acceso por una sola llave Costos Tiempo de procesamiento requerido para
	la aplicaci�n de la funci�n hash.
  \end{block}
\end{frame}

\begin{frame}
  \frametitle{Conclusi�n}
	El m�todo de b�squeda hash o por transformaci�n de clave aumenta la
	velocidad de b�squeda sin necesidad de que los elementos est�n
	previamente ordenados.
\end{frame}

\begin{frame}
  \frametitle{}
      
  �PREGUNTAS?
\end{frame}

\end{document}
