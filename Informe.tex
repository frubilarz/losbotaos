\documentclass[letterpaper,openright,12pt]{report}
\usepackage[spanish]{babel} % espanol
\usepackage[utf8]{inputenc} % acentos sin codigo
\usepackage{graphicx} % graficos
\usepackage{picture}

\begin{document}

\begin{titlepage}
\setlength{\unitlength}{1 cm} %Especificar unidad de trabajo
\begin{center}
\vspace*{-1in}
\begin{picture}(18,4)
%\begin{center}
\vspace*{0.18in}
\put(0,0){\includegraphics[width=3cm,height=4cm]{./.imagen/Logoutem.jpg}}
\put(11.5,0){\includegraphics[width=4cm,height=4cm]{./.imagen/iccilogocolor.png}}
%\includegraphics[width=4cm]{./.imagen/Logoutem.jpg}
%\end{center}
\end{picture}
\\[2cm]
\textbf{{\Huge Universidad Tecnológica Metropolitana}\\[1.5cm]
{\LARGE Escuela De informática}}\\[1.25cm]
{\LARGE \textbf{Busqueda Mediante Hashing}}\\[2.5cm]
{\large Integrantes:}\\
David Martínez Rivas\\
Felipe Canales Saavedra\\
Javier Reyes Gonzalez\\
Fernando Rubilar Zepeda\\
Profesor:\\
Alejandro Reyes\\[2cm]
Chile - \today
\end{center}

\end{titlepage}
\newpage
\begin{center}
\section{Introducción}

En informática, hash se refiere a una función o método para generar
claves o llaves que representan de manera casi unívoca a un
documento,registr, arichivo, etc., resumir o identificar un dato a
tráves de la probabilidad, utilizando una función hash o algoritmo hash.
Un has es el resultado de dicha función o algoritmo
\end{center}
\newpage
\section{Historia}

\textbf{Orígenes del término} El término hash proviene, aparentemente,
de la analogía con el significado estándar (en inglés) de dicha palabra
en el mundo real: picar y mezclar. Donald Knuth cree que H. P. Luhn,
empleado de IBM, fue el primero en utilizar el concepto en un memorándum
fechado en enero de 1953. Su utilización masiva no fue hasta después de
10 años. En el algoritmo SHA-1, por ejemplo, el conjunto de partida de
la función es dividido en palabras que son mezcladas entre sí utilizando
funciones matemáticas seleccionadas especialmente. Se hace que el rango
de valores que puede devolver la función sea de longitud fija: 160 bits
utilizando la adición modular.
\newpage
\section{Función hash}

Es una función para resumir o identificar probabilísticamente un gran
conjunto de información, dando como resultado un conjunto imagen finito
generalmente menor. Varían en los conjuntos de partida y de llegada y en
cómo afectan a la salida similitudes o patrones de la entrada

\subsection{Ventajas}

Se pueden usar los valores naturales de la llave, puesto que se traducen
internamente a direcciones fáciles de localizar Se logra independencia
lógica y física, debido a que los valores de las llaves son
independientes del espacio de direcciones No se requiere almacenamiento
adicional para los índices.

\subsection{Desventajas}

El archivo no esta clasificado No permite llaves repetidas Solo permite
acceso por una sola llave Costos Tiempo de procesamiento requerido para
la aplicación de la función hash
\end{document}
